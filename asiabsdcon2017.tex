\documentclass[sigconf]{acmart}

\usepackage{booktabs} % For formal tables


% Copyright
%\setcopyright{none}
%\setcopyright{acmcopyright}
%\setcopyright{acmlicensed}
%\setcopyright{rightsretained}
%\setcopyright{usgov}
%\setcopyright{usgovmixed}
%\setcopyright{cagov}
%\setcopyright{cagovmixed}


% DOI
\acmDOI{}

% ISBN
\acmISBN{}

%Conference
\acmConference[ASIABSDCON'17]{Aisa BSD Conference}{March 2017}{Tokyo, Japan} 
\acmYear{2017}
\copyrightyear{2017}

\acmPrice{}

\begin{document}
\title{Continuous Integration of The FreeBSD Project}
\subtitle{Build the daemon with daemon}

\author{Li-Wen Hsu}
\affiliation{The FreeBSD Project}
\email{lwhsu@FreeBSD.org}

\begin{abstract}
The FreeBSD project's continuous integration project started in the late 2013.
We use Jenkins automation server to build our continuous integration system.
It monitors the svn repository for new commits and triggers a new build of it.
In each build, the build machine compiles the latest code, creates disk image
and creates a virtual machine to run test suite. In the meantime, we collect
the compiler warnings and perform some further checks like clang analyzer. All
these information are published to the developers and users to improve the
quality of the FreeBSD project. In this paper, we describe the details of the
system implementation.
\end{abstract}

\keywords{Continuous Integration, FreeBSD, Jenkins}

\maketitle

\section{Introduction}

\section{Automatic Build and History of CI work in FreeBSD}

\section{Self Tests}

\section{Access to Results}

\section{Open Configurations}

\section{Other Integrations}

\section{Conclusion}

\section{Future Work}

\bibliographystyle{ACM-Reference-Format}
\bibliography{sigproc} 

\end{document}
